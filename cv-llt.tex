%% to generate this on visual code:
%% build latex project, if bib doesnt update, run pdflatex->bibtex->pdflatex*2 ONCE, then latexmk or build project again
\documentclass[a4paper,skipsamekey,11pt,english]{curve}

% Uncomment to enable Chinese; needs XeLaTeX
% \usepackage{ctex}


% Default biblatex style used for the publication list is APA6. If you wish to use a different style or pass other options to biblatex you can change them here. 
\PassOptionsToPackage{style=ieee,sorting=ydnt,uniquename=init,defernumbers=true}{biblatex}

% Most commands and style definitions are in settings.sty.
\usepackage{settings}

% If you need to further customise your biblatex setup e.g. with \DeclareFieldFormat etc please add them here AFTER loading settings.sty. For example, to remove the default "[Online] Available:" prefix before URLs when using the IEEE style:
\DefineBibliographyStrings{english}{url={\textsc{url}}}

%% Only needed if you want a Publication List
\addbibresource{own-bib.bib}

%% You can specify multiple names like this, especially if you have changed your name or if you need to highlight multiple authors. See items 6–9 in the example "Journal Articles" output.
\mynames{Angeles-Albores/David,
  Angeles\bibnamedelima{} Albores/David,
  Angeles-Albores/\bibnamedelimi{} D.,
  Angeles-Albores$^\dagger$/David,
  Clubes\bibnamedelima de\bibnamedelima Ciencia\bibnamedelima México/Workshop\bibnamedelima f.\bibnamedelima D.\bibnamelima B.
  }


% Change the fonts if you want
\ifxetexorluatex{} % If you're using XeLaTeX or LuaLaTeX
  \usepackage{fontspec} 
  %% You can use \setmainfont etc; I'm just using these font packages here because they provide OpenType fonts for use by XeLaTeX/LuaLaTeX anyway
  \usepackage[p,osf,swashQ]{cochineal}
  \usepackage[medium,bold]{cabin}
  \usepackage[varqu,varl,scale=0.9]{zi4}
\else % If you're using pdfLaTeX or latex
  \usepackage[T1]{fontenc}
  \usepackage[p,osf,swashQ]{cochineal}
  \usepackage{cabin}
  \usepackage[varqu,varl,scale=0.9]{zi4}
\fi

% Change the page margins if you want
% \geometry{left=1cm,right=1cm,top=1.5cm,bottom=1.5cm}

% Change the colours if you want
% \definecolor{SwishLineColour}{HTML}{00FFFF}
% \definecolor{MarkerColour}{HTML}{0000CC}

% Change the item prefix marker if you want
% \prefixmarker{$\diamond$}

%% Photo is only shown if "fullonly" is included
\includecomment{fullonly}
% \excludecomment{fullonly}


%%%%%%%%%%%%%%%%%%%%%%%%%%%%%%%%%%%%%%


\leftheader{%
  {\LARGE\bfseries\sffamily David Angeles-Albores, Ph.D.}

  \makefield{\faEnvelope[regular]}{\href{mailto:davidangelesalbores@gmail.com}{\texttt{davidangelesalbores@gmail.com}}}
  \makefield{\faLinkedin}
  {\href{http://www.linkedin.com/in/dangelesgenetics/}{\texttt{LinkedIn}}}
  %% Next line
  \makefield{\faGlobe}{\url{https://dangeles.github.io/}}

  % You can use a tabular here if you want to line up the fields.
}

\title{Curriculum Vitae}

\begin{document}
\makeheaders[c]

\makerubric{education}
\makerubric{employment}


% If you're not a researcher nor an academic, you probably don't have any publications; delete this line.
%% Sometimes when a section can't be nicely modelled with the \entry[]... mechanism; hack our own and use \input NOT \makerubric
%% Sometimes when a section can't be nicely modelled with the \entry[]... mechanism; hack our own
\newpage
\makerubrichead{Research Publications}

{\small \emph{{$^\dagger$} denotes equal contributions.}}

\nocite{*}

%% If you just want everything in one list
% \printbibliography[heading={none}]

\printbibliography[heading={subbibliography}, title={In Preparation, Press or
                   Revision}, keyword=prep, notkeyword=micropublication]

\printbibliography[heading={subbibliography}, title={Journal Articles},
                  type=article, notkeyword=micropublication, notkeyword=prep,
                  notkeyword=patent]

\printbibliography[heading={subbibliography}, title={Patents},
                  type=article, keyword=patent, notkeyword=prep]


\printbibliography[heading={subbibliography}, title={Conference Proceedings},
                   type=inproceedings]

\printbibliography[heading={subbibliography}, title={Books and Chapters},
                   filter={booksandchapters}]

\printbibliography[heading={subbibliography}, title={$\mu$Publications},
                   keyword=micropublication]


\makerubric{misc}


\end{document}