\begin{rubric}{Appointments}
\entry*[01/2022 - \textit{Present}]
	\textbf{Sr Computational Scientist}, Altos Labs\par
	I joined Altos Labs at launch. As one of its first bioinformaticians,
	I was a key contributor to its major lead rejuvenation candidate programs.
	I helped build out the genomics platforms for Altos Labs, and identified
	and initiated AI/ML research programs that are considered of top strategic
	importance. I developed end-to-end pipelines for analysis of Perturb-seq data,
	RNA-seq, ATAC-seq and pseudo-paired imaging/genomics data.
\entry*[07/2020 - \textit{Present}]
	\textbf{Visiting Scholar, Northwestern University}\par
	I research the effects of pheromone signaling on the lifespan, healthspan and sexual
	behaviors of \textit{C.~elegans} with Ilya Ruvinsky.
\entry*[03/2021 - 01/2022]
	\textbf{Senior Scientist I}, Rheos Medicines\par
	I developed precision biology methods using multimodal genomics measurements to
	identify responder and non-responder populations to candidate drugs.
	Rheos was unable to secure Series B funding in 2021 and closed its doors in 2022.
\entry*[11/2019 - 3/2021]
	\textbf{Computational Biologist II}, eGenesis\par
	I collaborated in the humanization of pig organs for transplantation into human patients
	by computationally designing a compendium of pig promoters that stably
	express genes ubiquitously or tissue-specificity in pig organs. I also
	developed methods to identify safe harbors resistant to epigenetic silencing.
	This work resulted in the transplantation of a pig kidney into a patient with
	end-stage renal disease who was able to leave the hospital without dialysis,
	and was featured in the 
	\href{https://www.nytimes.com/2024/03/21/health/pig-kidney-organ-transplant.html}{\texttt{\textbf{New York Times}}}.
\entry*[01/2019--11/2019]
		\textbf{Postdoctoral Associate, MIT},\par
		\textbf{Lab of Eric J. Alm}\par	
		I developed computational methods to deconvolve individual transcriptomes
		from metatranscriptomes collected from communities with a known bacterial
		composition without the need for any molecular barcodes. This enabled novel
		kinds of experiments to be carried out easily and cheaply.	
\end{rubric}
