\begin{rubric}{Appointments}
	\entry*[01/2022 - \textit{Present}]
		\textbf{Sr Computational Scientist}, Altos Labs,\par
		I perform synthetic biology research focusing on understanding homeostatic
		and proteostatic mechanisms by designing experiments and analyzing multi-modal
		datasets to generate testable hypotheses and identify causal genetic pathways
		that can be modulated to return cells to a healthy state from a diseased
		state.
\entry*[07/2020 - \textit{Present}]
	\textbf{Visiting Scholar}, Laboratory of Ilya Ruvinsky,\par
	\textbf{Northwestern University}\par
	I collaborate with the Ruvinsky lab on RNA-seq analyses of \textit{C.~elegans}
	biology.
\entry*[03/2021 - 01/2022]
	\textbf{Senior Scientist I}, Rheos\par
	At Rheos, I help design experiments using single-cell RNA-seq, ATAC-seq and
	mass spectroscopy and perform the data analaysis.\par
	I leverage artificial intelligence and machine learning to extract novel
	insights from genomics datasets and identify or verify promising drug targets.
\entry*[11/2019 - 3/2021]
	\textbf{Computational Biologist II}, eGenesis\par
	One of two founding members of the computational biology unit.\par
	Introduced ATAC-seq, scRNA-seq and scATAC-seq to eGenesis.\par
	I collaborated on the development of kidney dissociation protocols and used
	these protocols to create an atlas of the pig kidney cortex using scRNA-seq.\par
	I used the above methods to generate a compendium of promoters that stably
	express genes ubiquitously or with high tissue-specificy at desired levels with
	low burstiness. \textit{Patent pending}.

\entry*[01/2019--11/2019]
		\textbf{Postdoctoral Associate}, Laboratory of Eric J. Alm,\par
		\textbf{Massachusetts Institute of Technology}\par
		I developed methods for barcodeless, highly multiplexed RNA-seq of multiple
		bacterial species using a single library preparation protocol. \par
		My main project centered around recreating the vaginal microbiome \textit{in
		vitro} and dissecting the causal interactions between microbial species
		using a mixture of computational and experimental methods. I hoped to apply
		the algorithms I developed at Caltech (see below) to develop active learning
		models that combined iterative experimental minibatches with continuous
		computational analysis.\par
		\textit{I terminated my postdoctoral research prematurely due to a major
		personal loss, unrelated to my work at MIT.}

\entry*[11/2018--01/2019]
		\textbf{Postdoctoral Fellow}, Labs of Paul W. Sternberg and Matt Thomson,\par
		\textbf{California Institute of Technology}\par
		I used this period immediately following my defense to finish
		projects with my Ph.D. advisor, Paul Sternberg, and a member of my
		committee, Matt Thomson, before moving on to my postdoctoral position
		at MIT in January, 2019. I developed active learning algorithms for analysis
		of CRISPR screens in mammalian cells, using transcriptomes as phenotypes.

\end{rubric}
